\documentclass{article}

\usepackage{mypckg2}

\begin{document}
	\section{Duale Paare, schwache und duale Topologien}
	
	\begin{Def}{Duales Paar}{}
		Seien $X,Y$ $\KK$-Vektorräume und $\p{\bullet}{\bullet}\colon X\times Y\to \KK$ eine bilineare Abbildung. Sind die Familien $\{x\mapsto \p{x}{y}\}_{y\in Y}$ und $\{y\mapsto \p{x}{y}\}_{x\in X}$ punktetrennend, das heißt
		\[\forall x\in X\setminus\{0\}\;\exists y\in Y:\;\;\p{x}{y}\ne 0\;\;\te{und }\;\;\forall y\in Y\setminus\{0\}\;\exists x\in X:\;\;\p{x}{y}\ne 0\,,\]
		so nennt man $(X,Y,\p{X}{Y})$ ein \textbf{duales Paar}.
	\end{Def}
	
	\begin{Bemerkung}{}{}
		Man kann die Elemente von $X$ bzw. $Y$ als lineare Funktionale auf $Y$ bzw. $X$ auffassen, via
		\[\ell_{y}\colon x\mapsto \ell_{y}(x)=\p{x}{y}\]
		bzw.
		\[\ell_{x}\colon y\mapsto \ell_{x}(y)=\p{x}{y}\,.\]
	\end{Bemerkung}
	
	\begin{Beispiel}{}{}
		\begin{itemize}
			\item[(i)] Ist $X$ ein lokalkonvexer Hausdorff-Raum, dann ist $(X,X')$ ein duales Paar, mit
			\[(x,f)\mapsto \p{x}{f}:=f(x)\,.\]
			Nach Hahn Banach ist die Familie $X'$ punktetrennend und $X$ ist trivialerweise punktetrennend.
			\item[(ii)] Genauso ist $(X',X)$ ein duales Paar.
		\end{itemize}
	\end{Beispiel}
	
	\begin{Def}{$\sigma(X,Y)$-Topologie}{}
		Sei $(X,Y)$ ein duales Paar. Die \textbf{$\sigma(X,Y)$-Topologie} auf $X$ ist die lokalkonvexe Toplogie, die von den Halbnormen $(x\mapsto|\p{x}{y}|)_{y\in Y}$ erzeugt wird.
	\end{Def}
	
	\begin{Bemerkung}{}{}
		\begin{itemize}
			\item[(i)] Die $\sigma(X,Y)$-Topologie ist Hausdorff, weil wir in der Definition von \enquote{dualem Paar} punktetrennend gefordert haben.
			\item[(i)] Die $\sigma(X,Y)$-Topologie ist die gröbste Topologie auf $X$, sodass alle $\ell_{y}\colon X\to \KK$ stetig sind (siehe z.B. Satz VIII.3.6 oder Übungsaufgabe in Funkana1?)
		\end{itemize}
	\end{Bemerkung}
	
	Frage: Was ist der topologische Dualraum von $X$ mit der $\sigma(X,Y)$-Topologie?\\
	Antwort: Es ist genau $Y$!\\
	\\
	Für den Beweis benötigen wir folgendes Lemma:
	\begin{Lemma}{}{}
		Sei $X$ ein $\KK$-Vektorraum und $\ell,\ell_1,\ldots,\ell_n\colon X\to \KK$ lineare Funktionale. Setze $N:=\cap_{i=1}^n\ker(\ell_i)$, dann sind äquivalent:
		\begin{itemize}
			\item[(i)] $\ell\in \spann(\ell_1,\ldots,\ell_n)$
			\item[(ii)] $\exists M>0\;\forall x\in X:\;\;|\ell(x)|\le M\max_{1\le i\le n}|\ell(x_i)|$
			\item[(iii)] $N\subseteq \ker(\ell)$.
		\end{itemize} 
	\end{Lemma}
	
	\begin{Beweis}
		(i)$\implies$(ii): Sei $\ell=\sum \alpha_i\ell_i$, dann gilt
		\[\ell(x)|\le \sum |\alpha_i\ell_i(x)|\le \max \alpha_i\max_{|\ell_i(x)|}\,.\]
		(ii)$\implies$(iii): Sei $x\in N$, dann gilt
		\[|\ell(x)|\le m\max_{1\le i\le n}|\ell(x_i)|=0\,.\]
		(iii)$\implies$ (i):
	\end{Beweis}
	
	Daraus folgt sofort das folgende Korollar:
	
	\begin{Kor}{}{}
		Sei $(X,Y)$ ein duales Paar. Dann ist
		\[(X,\sigma(X,Y))'=Y\,,\]
		in dem Sinn, dass jedes lineares, $\sigma(X,Y)$-stetiges Funktional auf $X$ von der Form $x\mapsto \p{x}{y}$ für ein bestimmtes $y$ ist und dass jedes Funktional dieser Form linear und $\sigma(X,Y)$ stetig ist, also:
		\[(X,\sigma(X,Y))'=\left\{x\mapsto \p{x}{y}\,\big|\, y\in Y\right\}\]
	\end{Kor}
	
	\begin{Beweis}
	 $\subseteq$: Sei $f\colon X\to \KK$ linear und $\sigma(X,Y)$-stetig. Dann gibt es endlich viele Halbnormen, sodass
	 \[|f(x)|\le M\max_{i}|p_i(x)|\]
	 Die Halbnormen sind gerade gegeben durch $y$, also
	 \[f(x)|\le M\max_i |\p{x}{y}|\,.\]
	 Nach dem Lemma also $f\in \spann(\ell_{y_1},\ldots,\ell_{y_n})$, also $f=\ell_{\sum \alpha_iy_i}$.\\
	 $\subseteq$: Sei $y\in Y$, dann ist $\ell_y$ linear und $|\ell_y|$ eine Halbnorm auf $X$, die die lct erzeugt, also stetig.
	\end{Beweis}
	
	\begin{Def}{}{}
		Sei $(X,Y)$ ein duales Paar. Eine lokalkonvexe Topologie $\tau$ auf $X$ heißt \textbf{$\p{X}{Y}$-duale Topologie}, falls 
		\[(X,\tau)'=Y\,,\]
		also die linearen, $\tau$-stetigen Funktionale genau die von der Form
		\[x\mapsto\p{x}{y}\]
		für $y\in Y$ sind.
	\end{Def}
	
	Nach dieser Definition ist $\sigma(X,Y)$ die gröbste $\p{X}{Y}$-duale Topologie. Eine Charakterisierung aller $\p{X}{Y}$-dualen Topologien gibt der Satz von Mackey-Ahrens, den wir später besprechen werden.
	
	\section{Polare Mengen und der Bipolarensatz}
	
	\begin{Def}{}{}
		Sei $(X,Y)$ ein duales Paar und $A\subseteq X$, $B\subseteq Y$. Dann ist die \textbf{Polare} von $A$ gegeben durch
		\[A^{\circ}:=\left\{y\in Y\,\big|\,\sup_{x\in A}\Rea(\p{x}{y})\le 1\right\}\tag{Teilmenge von $Y$}\]
		und die Polare von $B$ durch
		\[B^{\circ}:=\left\{x\in X\,\big|\,\sup_{y\in B}\Rea(\p{x}{y})\le 1\right\}\tag{Teilmenge von $X$}\]
	\end{Def}
	
	{\fontencoding{U}\fontfamily{futs}\selectfont\char 49\relax} Oft wird auch $|\p{x}{y}|\le 1$ statt $\Rea(\p{x}{y})$ gefordert.
	

	
	\begin{Lemma}{Eigenschaften von Polaren}{1}
		Sei $(X,Y)$ ein duales Paar und $A\subseteq X$. Dann gilt:
		\begin{itemize}
			\item[(i)] $0\in A^{\circ}$, $A^{\circ}$ ist konvex und abgeschlossen bzgl. $\sigma(Y,X)$ und $A^{\circ}=\overline{\mathrm{conv}(A)}^{\circ}$.
			\item[(ii)] $A\subseteq A^{\circ\circ}$ und $A\subseteq B\implies B^{\circ}\subseteq A^{\circ}$.
			\item[(iii)] Sei $I$ eine Indexmenge und $(A_i)_{i\in I}\subseteq X^I$. Dann gilt
			\[\left(\bigcup_{i\in I} A_i\right)^{\circ}=\bigcap_{i\in I}A_i^{\circ}\,.\]
			\item[(iv)] Ist $A$ kreisförmig, dann ist $A^{\circ}=\{y\in Y\,|\, \sup_{x\in A}|\p{x}{y}|\le 1\}$.
			\end{itemize}
	\end{Lemma}
	
	\begin{proof}
		Aus dem Kopf
	\end{proof}
	
		\begin{Beispiel}{}{}
		\begin{itemize}
			\item[(i)] Ist $X$ ein normierter Raum, dann ist
			\[(B_1^X)^{\circ}=B_1^{X'}\,.\]
			Beweis: klar
		\end{itemize}
	\end{Beispiel}
	
	
	
	\begin{Satz}{Bipolarensatz}{}
		Sei $(X,Y)$ ein duales Paar und $A\subseteq X$. Dann ist 
		\[A^{\circ\circ}=\overline{\mathrm{conv}(A\cup \{0\})}\,.\]
	\end{Satz}
	
	\begin{proof}
		\enquote{$\supseteq$}: Note that, by the previous lemma:
		\[(A\cup\{0\})^{\circ}=A^{\circ}\cap\underbrace{\{0\}^{\circ}}_{Y}=A^{\circ}\implies (A\cup\{0\})^{\circ\circ}=A^{\circ\circ}\,.\]
		Hence:
		\[A^{\circ\circ}=(A\cup \{0\})^{\circ\circ}=\overline{\mathrm{conv}(A\cup \{0\})}^{\circ\circ}\supseteq \overline{\mathrm{conv}(A\cup \{0\})}\,.\]
		\enquote{$\subseteq$}: Angenommen, $\exists x\in A^{\circ\circ}\setminus\overline{\mathrm{conv}(A\cup \{0\})}$. Da $V:=\overline{\mathrm{conv}(A\cup \{0\})}$ konvex und $\sigma(X,Y)$-abgeschlossen ist (Abschluss von konvexen Mengen ist konvex), gibt es nach Hahn-Banach ein $\sigma(X,Y)$-stetiges Funktional, dass $x$ und $V$ trennt. Wegen $(X,\sigma(X,Y))'=Y$, gibt es also ein $y\in Y$ und ein $\ve>0$, sodass:
         \begin{align*}
         &\forall v\in V:\;\; \Rea(\p{x}{y})+\ve\le \Rea(\p{v}{y})\\
         \iff& \forall v\in V:\;\;\Rea(\p{x}{-y})-\ve\ge \Rea(\p{v}{-y})\\
         \implies& \sup_{v\in V}\Rea(\p{v}{-y})\le \Rea(\p{x}{-y})-\ve<\Rea(\p{x}{-y})-\frac{\ve}{2} \\
         \iff& \underbrace{\underbrace{\sup_{v\in V}\Rea(\p{v}{-y})}_{\ge 0,\te{da }0\in V}+\frac{\ve}{2}}_{=:~C>0}<\Rea(\p{x}{-y})
         \end{align*}
         Definiere $\tilde y:=-\frac{y}{C}$. Dann gilt
         \[1<\Rea(\p{x}{\tilde{y}})\tag{1}\label{1}\]
         und 
         \[\sup_{v\in V} \Rea(\p{v}{\tilde y})= \frac{1}{C}\sup_{w\in V}\Rea(\p{w}{-y})\le 1\,.\tag{2}\label{2}\] 
         \eqref{2} impliziert $\tilde{y}\in V^{\circ}\subseteq A^{\circ}$ (da $A\subseteq V$). Damit folgt jedoch aus \eqref{1}, dass $x\notin A^{\circ\circ}$ \lightning. 
         
	\end{proof}
	
	\begin{Kor}{Charakterisierung abgeschlossener konvexer Mengen}{}
		Sei $(X,Y)$ ein duales Paar und $A\subseteq X$ konvex mit $0\in A$. $A$ ist genau dann abgeschlossen bzgl. $\sigma(X,Y)$, wenn $A$ eine Polare ist, also
		\[A\te{ist abgeschlossen bzgl. }\sigma(X,Y)\iff \exists B\subseteq Y:\;\; B^{\circ}=A\,.\]
	\end{Kor}
	
	\begin{proof}
		\enquote{$\implies$} Wähle $B:=A^{\circ}$, dann gilt:
		\[B^{\circ}=A^{\circ\circ}\overset{\ms{\text{Bidualensatz}\\\big\downarrow}}{=}\overline{\mathrm{conv}(A\cup \{0\})}=A\]
		\enquote{$\impliedby$} Folgt aus Lemma \ref{Lemma:1}(i), angewendet auf das duale Paar $(Y,X)$.
	\end{proof}
	
	\section{Der Satz von Alaoglu-Bourbaki}
	
	Hier betrachten wir das duale Paar $(X,X')$ eines lokalkonvexen Raums mit seinem Dualraum.
	
	\begin{Satz}{Alaoglu-Bourbaki}{}
		Sei $U\subseteq X$ eine Nullumgebung. Dann ist $U^{\circ}$ kompakt bezüglich $\sigma(X',X)$.
	\end{Satz}
	
	\begin{proof}
		Vorüberlegung: Tychonoff gibt eine Kompaktheitsbedingung in der Produkttopologie auf $\KK^X$.\\
		Sei $\KK^X$ die Menge aller Funktionale auf $X$, dann ist offensichtlich $X'\subseteq X$. Sei $\tau_p$ die Produkttopologie auf $\KK^X$ und $\KK^X|_{X'}$ die Teilraumtopologie auf $X'$. Die Produktopologie ist die gröbste Topologie, bezüglich der alle kanonischen Projektionen
		\[\pi_x\colon\begin{array}{clc}
			\KK^X&\to&\KK\\
			f&\mapsto& f(x)
		\end{array}\,,\]
		stetig sind. Dies sind gerade die Auswertungsfunktionale. Somit ist $\tau_p$ die Topologie der punktweisen Konvergenz, denn falls $f_n\to f\in \KK^X$ wrt. $\tau_p$, dann gilt auch $f_n(x)=\pi_x(f_n)\to \pi_x(f)=f(x)$ für jedes $x$.
		\begin{itemize}
			\item $\KK^X,\tau_p$ ist ein topologischer Vektorraum, d.h. $+\colon \KK^X\times \KK^X\to \KK^X$ und $\cdot\colon \KK\times \KK^X\to\KK^X$ sind stetig wrt. $\tau_p$.
			\item Somit ist auch $(X',\tau_p|_{X'})$ ein topologischer Vektorraum, denn $+\colon X'\times X'\to \KK^X$ und $\cdot \colon \KK\times X'\to \KK^X$ sind stetig wrt. $\tau_p|_{X'}$.
			\item Außerdem sind die eingeschränkten kanonischen Projektionen stetig wrt. $\tau_p|_{X'}$:
			\[\pi_x|_{X'}\colon \begin{array}{clc}
				X'&\to&\KK\\
				f&\mapsto&f(x)
			\end{array}\]			
		\end{itemize}
		Nach Definition ist die schwach-$*$-Topologie ($\sigma(X',X)$) auf $X'$ die gröbste Vektorraumtopologie, sodass alle Auswertungsfunktionale stetig sind, daher muss $\tau_p|_{X'}$ feiner sein als $\sigma(X',X)$, d.h.
		\[\sigma(X',X)\subseteq \tau_p|_{X'}\,.\]
		Daraus folgt, dass die Identität
		\[\id \colon (X',\tau_p|_{X'})\to (X',\sigma(X',X))\]
		stetig ist. Das heißt, falls eine Menge $A\subseteq X'$ kompakt bezüglich $\tau_p|_{X'}$ ist, dann auch bezüglich $\sigma(X',X)$.\\
		\\
		Sei nun $U\subseteq X$ eine Nullumgebung, dann gibt es eine absolutkonvexe offene Menge $V\subseteq U$, denn es gibt eine Nullumgebungsbasis aus absolutkonvexen Mengen. Außerdem ist $V$ als Nullumgebung absorbierend: Sei $x\in X$, dann ist
		\[g_x\colon \begin{array}{clc}
			\KK&\to&X\\
			\lambda&\mapsto&\lambda x
		\end{array}\]
		stetig. Also ist $g^{-1}(V)$ eine Nullumgebung in $\KK$, das heißt $\exists \ve_x>0$ sodass $\forall 0<\ve\le\ve_x$: $\ve\in g^{-1}(V)$, was äquivalent ist zu $\ve x\in V\iff x\in \ve^{-1} V$. Wähle
		\[\lambda_x:=\left\{\begin{array}{cl}
			1,&x\in V\\
			\ve_x,& x\notin V
			\end{array}
			\right.\]
		und $K_x:=\overline{B_{\lambda_x^{-1}}(0)}\subseteq \KK$, was offensichtlich kompakt ist. Nach Tychonoff ist
		\[K:=\bigtimes_{x\in X}K_x=\{f\colon X\to \KK\,|\,\forall x\in X:\;\;f(x)\in K_x\}\]
		kompakt in $\tau_p$.\\
		\\
		Wir zeigen $V^{\circ}\subseteq K$. Sei $f\in V^{\circ}$. Dann gilt, weil $V$ absolut konvex ist:
		\[\forall x\in V:\;\;|\p{f}{x}|\le 1\]
		Sei nun $x\in X$ beliebig, dann ist $\lambda_xx\in V$ nach Konstruktion und somit
		\[|\p{f}{\lambda_xx}|\le 1\iff |\p{f}{x}|\le \frac{1}{\lambda_x}\iff f(x)\in K_x\,.\]
		
		Zu guter Letzt zeigen wir, dass $V^{\circ}$ abgeschlossen in $\tau_p|_{K}$ ist. Sei dazu $\{f_{\alpha}\}_{\alpha\in I}$ ein Netz in $V^{\circ}$ mit $\lim_{\alpha\in I}f_{\alpha}=f$.
		\begin{itemize}
			\item $f$ ist linear, weil Vektoraddition und Skalarmultiplikation stetig sind.
			\item $f\colon X\to \KK$ ist stetig, denn:
			\begin{align*}
			&f\colon X\to \KK\te{stetig}\\
			\iff& \exists M>0\,\exists p_1,\ldots,p_n\,\forall x\in X:\;\;|f(x)|\le M\max_{1\le j\le n}|p_j(x)|\\
			\iff & |f|\colon X\to \RR_+\te{stetig}\\
			\overset{\ms{|f|\te{Halbnorm}\\\downarrow}}{\iff}& \{x\in X\,|\, |f(x)|\le 1\} \te{Nullumgebung}
			\end{align*}
			Wir zeigen $V\subseteq \{x\in X\,|\, |f(x)|\le 1\}$. Sei $x\in V$, dann gilt für alle $\alpha\in I$:
			\[1\ge |\p{f_{\alpha}}{x}|\]
			und somit auch $|\p{f}{x}|\le 1$.
		\end{itemize}
		Somit ist $f\in X'$. Das heißt, $f_{\alpha}$ ist ein konvergentes Netz in $X'$ bezüglich $\tau_p|_{X'}$. Weil die Identität stetig ist, ist auch $\lim_{\alpha\in I}f_{\alpha}=f$ bezüglich $\sigma(X',X)$. Somit ist $f\in \overline{V^{\circ}}^{\sigma(X',X)}=V^{\circ}$.\\
		\\
		Somit ist $V^{\circ}$ eine abgschlossene Teilmenge des Kompaktums $K$ und somit selbst kompakt in $\tau_p|_{K}$. Also auch in $\tau_p$ und auch in $\tau_p|_{X'}$. Nach unserer Vorüberlegung, ist $V^{\circ}$ also auch kompakt in $\sigma(X',X)$. Nun ist ja $V\subseteq U$, also $U^{\circ}\subseteq V^{\circ}$ und $U^{\circ}$ ist abgschlossen in $\sigma(X',X)$ also ist auch $U^{\circ}$ kompakt in $\sigma(X',X)$.		
	\end{proof}
	
	\begin{Kor}{Banach-Alaoglu}{}
		Ist $X$ ein normierter Raum, dann ist $B_1(0)\subseteq X'$ weak-$*$-kompakt.
	\end{Kor}
	
	\begin{proof}
		Folgt aus Alaoglu-Bourbaki, mit $U=B_1(0)\subseteq X$ und Beispiel \ref{}.
	\end{proof}
	
	
	\section{Mackey topologies and the Mackey-Ahrens theorem}
	
	Wir wissen ja bereits, dass die $\sigma(X,Y)$-Topologie die gröbste duale Topologie ist, das heißt die gröbste Topologie, sodass
	\[X'\ni f\leftrightarrow \ell_y\,.\]
	Für eine gröbere Topologie gibt es $y\in Y$, sodass $\ell_y$ nicht stetig ist.\\
	Aber man kann auch die Gegenteilige Frage stellen: Welches ist die feinste duale Topologie, also welches ist die feinste lokalkonvexe Toplogie, sodass es keine stetigen linearen Funktionale gibt, die nicht von der Form $\ell_y$ sind?\\
	\\
	Dies ist gerade die Mackey-Topologie $\tau(X,Y)$, die gerade die lokalkonvexe Topologie ist, die von den Halbnormen
	\[\{x\mapsto \sup_{y\in C}|\p{x}{y}|\}_{C\subseteq Y\te{kompakt und konvex}}\]
	erzeugt wird.\\
	\\
	\begin{Satz}{Mackey-Ahrens}{}
		Sei $(X,Y)$ ein duales Paar und $\tau$ eine $\p{X}{Y}$-duale Topologie auf $X$. Dann gilt
		\[\sigma(X,Y)\subseteq \tau\subseteq \tau(X,Y)\,.\]
	\end{Satz}
	
	Es gibt einige schöne Resultate:
	\begin{itemize}
		\item Ein lokalkonvexer Raum ist reflexiv, gdw. er die Mackey Topologie trägt.
		\item Jeder Frechet Raum trägt die Mackey topology.
		\item Insbesondere ist also der Schwartzraum reflexiv.
		\item Im Schwartzraum ist jede abgschlossene, beschränkte Menge kompakt.
	\end{itemize}
	
	
	
	
\end{document}